%% start of file `template.tex'.
%% Copyright 2006-2013 Xavier Danaux (xdanaux@gmail.com).
%
% This work may be distributed and/or modified under the
% conditions of the LaTeX Project Public License version 1.3c,
% available at http://www.latex-project.org/lppl/.


\documentclass[10pt,a4paper,sans]{moderncv}        % possible options include font size ('10pt', '11pt' and '12pt'), paper size ('a4paper', 'letterpaper', 'a5paper', 'legalpaper', 'executivepaper' and 'landscape') and font family ('sans' and 'roman')
% moderncv themes
\moderncvstyle{classic}                             % style options are 'casual' (default), 'classic', 'oldstyle' and 'banking'
\moderncvcolor{blue}                               % color options 'blue' (default), 'orange', 'green', 'red', 'purple', 'grey' and 'black'
%\renewcommand{\familydefault}{\sfdefault}         % to set the default font; use '\sfdefault' for the default sans serif font, '\rmdefault' for the default roman one, or any tex font name
%\nopagenumbers{}                                  % uncomment to suppress automatic page numbering for CVs longer than one page

% character encoding
\usepackage[utf8]{inputenc}                       % if you are not using xelatex ou lualatex, replace by the encoding you are using
%\usepackage{CJKutf8}                              % if you need to use CJK to typeset your resume in Chinese, Japanese or Korean
% adjust the page margins
            
\usepackage[scale=0.85]{geometry}

%\setlength{\hintscolumnwidth}{3cm}                % if you want to change the width of the column with the dates
%\setlength{\makecvtitlenamewidth}{10cm}           % for the 'classic' style, if you want to force the width allocated to your name and avoid line breaks. be careful though, the length is normally calculated to avoid any overlap with your personal info; use this at your own typographical risks...
% personal data
\name{Song}{Cheng}
\title{Assistant Researcher}                               % optional, remove / comment the line if not wanted
\address{Center of Quantum Computing}{Peng Cheng Laboratory}{Shenzhen China}% optional, remove / comment the line if not wanted; the "postcode city" and and "country" arguments can be omitted or provided empty
\phone[mobile]{(+86)~18500048192}                   % optional, remove / comment the line if not wanted                 
\email{chengs@pcl.ac.cn}                               % optional, remove / comment the line if not wanted             % optional, remove / comment the line if not wanted
%\extrainfo{Birth: May, 24, 1992} %\photo[64pt][0.4pt]{picture}                       % optional, remove / comment the line if not wanted; '64pt' is the height the picture must be resized to, 0.4pt is the thickness of the frame around it (put it to 0pt for no frame) and 'picture' is the name of the picture file
%\quote{Some quote}                                 % optional, remove / comment the line if not wanted

% to show numerical labels in the bibliography (default is to show no labels); only useful if you make citations in your resume
%\makeatletter
%\renewcommand*{\bibliographyitemlabel}{\@biblabel{\arabic{enumiv}}}
%\makeatother
%\renewcommand*{\bibliographyitemlabel}{[\arabic{enumiv}]}% CONSIDER REPLACING THE ABOVE BY THIS

% bibliography with mutiple entries
%\usepackage{multibib}
%\newcites{book,misc}{{Books},{Others}}
%----------------------------------------------------------------------------------
%            content
%----------------------------------------------------------------------------------
\begin{document}
%\begin{CJK*}{UTF8}{gbsn}                          % to typeset your resume in Chinese using CJK
%-----       resume       ---------------------------------------------------------
\makecvtitle

\section{Education}
\cventry{2014,9-2019,6}{Ph.D in Theoretical physics}{Institute of Physics, Chinese Academy of Sciences}{Beijing, China}{}{Supervisors: Prof. Tao Xiang and Prof. Lei Wang}
\cventry{2010.9-2014.6}{B.S. in Physics}{Sichuan University}{Chengdu, China}{}{}  % arguments 3 to 6 can be left empty

\section{Skills}
\cvlistitem{Tranditional Tensor Networks algorithms (Written: DMRG, TEBD, TRG, SRG, HOTRG, HOSRG, TNR, loop-TNR. Well understood: MERA, CTMRG, Fermion PEPS, PESS)}
\cvlistitem{Machine Learning models (neural nets., graphical models, autoregressive models, flow models, etc.) and frameworks (tensorflow/pytorch)}
\cvlistitem{Some TN machine learning model, such as the MPS for image classification/generation and the Tree TN for image generation.}
\cvlistitem{Python/Matlab language}

\section{Interests}
\cvlistitem{Currently, I'm interest in both the traditional tensor network algorithms (on many body physics) and its application on machine learning and quantum computing problems.} %I care about the information/entanglement/entropy in both tensor network models and traditional machine learning models. }
%\cvitem{ Tensor Networks}{Description}
%\cvitem{Machine learning}{Description}
%\cvitem{Quantum computation}{Description}


%\section{Experience}
%\cvitem{2010--2014}{At the undergraduate level, I participated in the Ministry of Education's top talents program. This program selects outstanding students from the science colleges of 21 top universities in China each year to encourage them engaging in fundamental scientific research. In this level, I was well trained in a very classic theoretical physics style.}
%\cvitem{2014--2016}{After graduating, I joined the group of Prof. Tao Xiang. In the first two years, with the help of Prof. Zhiyuan Xie, I was focus on optimizing the accuracy of the tensor RG algorithm on the antiferromagnetic J1-J2 model. I have tried algorithms such as TEBD, TRG, SRG, HOTRG, HOSRG, etc. At the same time, I also cooperate with Prof. Xie and Prof. Xiang on  studying and improving the TNR and loop-TNR algorithm. During this time, I received a very specific and professional training on the tensor network numerical algorithm.}
%\cvitem{2017}{At the beginning of my third year, as we started to noticed the connection between tensor network and machine learning, I started to cooperate with Prof. Lei Wang and transferred my research to machine learning. Our first result is theoretically proved the exact mapping between RBM and MPS[3]. \newline 
%Our second result is to examine the feasibility of applying tensor network to machine learning problems from the perspective of information theory, and also proposing the extension of Boltzmann machines: Born machine[2]. \newline
%In the collaboration with Prof. Wang, I learned about Monte Carlo, machine learning algorithms, data structures and algorithms, scientific computing programming paradigm, etc. }
%\cvitem{2018--present}{Since 2018, I started working with Prof. Lei Wang and Prof. Pan Zhang to find a more powerful generative model than the previous MPS. Our choice is tree tensor network. Now we have obtained the current best result of the tensor network generative modeling. The article is in preparation[1]. During this time, I learned many statistical inference, spin glass, GPU parallelism and experience of programming from Pan.}

%\section{Awards}
%\cvitem{ 2017}{Director Scholarship of the Institute of Physics, Chinese Academy of Sciences.}
%\cvitem{2012, 2013 and 2014}{Top Talents Scholarship on Fundamental Disciplines of the Ministry of Education of China.}

\section{Publications }
\cvlistitem{total citation: 129}
\cventry{[1]}{\href{https://journals.aps.org/prb/abstract/10.1103/PhysRevB.99.155131}{Tree Tensor Networks for Generative Modeling}}{}{}{}{\textbf{S. Cheng}, L. Wang, T. Xiang, P. Zhang, Physical Review B,99.155131}
\date{}
\cventry{[2]}{ \href{https://www.mdpi.com/1099-4300/20/8/583}{Information perspective to probabilistic modeling: Boltzmann machines versus born machines}}{}{}{}{\textbf{S. Cheng}, J. Chen, L. Wang, Entropy 2018, 20, 583.}
\cventry{[3]}{\href{https://journals.aps.org/prb/abstract/10.1103/PhysRevB.97.085104}{Equivalence of restricted Boltzmann machines and tensor network states}}{}{}{}{J. Chen, \textbf{S. Cheng}, H. Xie, L. Wang, T. Xiang, Physical Review B, 97 (8), 085104.}
\cventry{[4]}{\href{https://arxiv.org/abs/1904.06194}{Compressing deep neural networks by matrix product operators}}{}{}{}{Z. F. Gao, \textbf{S. Cheng (co-author)}, R. Q. He, Z. Y. Xie, H. H. Zhao, Zhong Y. Lu, and T. Xiang, arXiv:1904.06194}
\cventry{[5]}{\href{http://iopscience.iop.org/article/10.1088/0256-307X/34/5/050503}{Phase Transition of the q-State Clock Model: Duality and Tensor Renormalization}}{}{}{}{Chen, J., Liao, H.-J., Xie, H.-D., Han, X.-J., Huang, R.-Z., \textbf{Cheng, S.}, et al.\ 2017, Chinese Physics Letters, 34, 050503.}

\section{Conferences}
\cvitem{1}{\textbf{Tree Tensor Network for Generative Modeling} \newline
The APS March Meeting, Boston, 2019.}
\cvitem{2}{\textbf{Born machine: generative modeling using tensor network states.} \newline
The 8th Workshop on Quantum Many-Body Computation, Huangzhou, 2018, Invited talk.}
\cvitem{3}{\textbf{A brief review on the application of tensor networks to the machine Learning problem.} \newline
Workshop on the Statistical Physics and Machine Learning, Anqing, 2018, Invited talk.}
\cvitem{4}{\textbf{Pixel Correlation and Mutual Information: Implication to Unsupervised Learning.} \newline
 Fall Meeting of the China Physics Society, Chengdu, 2017, Invited talk.} 

%\begin{cvcolumns}
%  \cvcolumn{Category 1}{\begin{itemize}\item Person 1\item Person 2\item Person 3\end{itemize}}
%  \cvcolumn{Category 2}{Amongst others:\begin{itemize}\item Person 1, and\item Person 2\end{itemize}(more upon request)}
 % \cvcolumn[0.5]{All the rest \& some more}{\textit{That} person, and \textbf{those} also (all available upon request).}
%\end{cvcolumns}

% Publications from a BibTeX file without multibib
%  for numerical labels: \renewcommand{\bibliographyitemlabel}{\@biblabel{\arabic{enumiv}}}% CONSIDER MERGING WITH PREAMBLE PART
%  to redefine the heading string ("Publications"): \renewcommand{\refname}{Articles}
\nocite{*}
\bibliographystyle{plain}
\bibliography{publications}                        % 'publications' is the name of a BibTeX file

% Publications from a BibTeX file using the multibib package
%\section{Publications}
%\nocitebook{book1,book2}
%\bibliographystylebook{plain}
%\bibliographybook{publications}                   % 'publications' is the name of a BibTeX file
%\nocitemisc{misc1,misc2,misc3}
%\bibliographystylemisc{plain}
%\bibliographymisc{publications}                   % 'publications' is the name of a BibTeX file

\clearpage
%-----       letter       ---------------------------------------------------------
% recipient data
%\recipient{Company Recruitment team}{Company, Inc.\\123 somestreet\\some city}
%\date{January 01, 1984}
%\opening{Dear Sir or Madam,}
%\closing{Yours faithfully,}
%\enclosure[Attached]{curriculum vit\ae{}}          % use an optional argument to use a string other than "Enclosure", or redefine \enclname
%\makelettertitle
%
%Lorem ipsum dolor sit amet, consectetur adipiscing elit. Duis ullamcorper neque sit amet lectus facilisis sed luctus nisl iaculis. Vivamus at neque arcu, sed tempor quam. Curabitur pharetra tincidunt tincidunt. Morbi volutpat feugiat mauris, quis tempor neque vehicula volutpat. Duis tristique justo vel massa fermentum accumsan. Mauris ante elit, feugiat vestibulum tempor eget, eleifend ac ipsum. Donec scelerisque lobortis ipsum eu vestibulum. Pellentesque vel massa at felis accumsan rhoncus.
%
%Suspendisse commodo, massa eu congue tincidunt, elit mauris pellentesque orci, cursus tempor odio nisl euismod augue. Aliquam adipiscing nibh ut odio sodales et pulvinar tortor laoreet. Mauris a accumsan ligula. Class aptent taciti sociosqu ad litora torquent per conubia nostra, per inceptos himenaeos. Suspendisse vulputate sem vehicula ipsum varius nec tempus dui dapibus. Phasellus et est urna, ut auctor erat. Sed tincidunt odio id odio aliquam mattis. Donec sapien nulla, feugiat eget adipiscing sit amet, lacinia ut dolor. Phasellus tincidunt, leo a fringilla consectetur, felis diam aliquam urna, vitae aliquet lectus orci nec velit. Vivamus dapibus varius blandit.
%
%Duis sit amet magna ante, at sodales diam. Aenean consectetur porta risus et sagittis. Ut interdum, enim varius pellentesque tincidunt, magna libero sodales tortor, ut fermentum nunc metus a ante. Vivamus odio leo, tincidunt eu luctus ut, sollicitudin sit amet metus. Nunc sed orci lectus. Ut sodales magna sed velit volutpat sit amet pulvinar diam venenatis.
%
%Albert Einstein discovered that $e=mc^2$ in 1905.
%
%\[ e=\lim_{n \to \infty} \left(1+\frac{1}{n}\right)^n \]
%
%\makeletterclosing
%
%%\clearpage\end{CJK*}                              % if you are typesetting your resume in Chinese using CJK; the \clearpage is required for fancyhdr to work correctly with CJK, though it kills the page numbering by making \lastpage undefined
\end{document}


%% end of file `template.tex'.
